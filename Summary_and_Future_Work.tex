\section{Future Work}\label{sec:summary}

We have established that individual self-assessment ratings should be used as a last resort. We have also shown that club-assessments ratings are more predictive of true skill level than captain-assessment ratings, which are themselves much better than self-assessment ratings. We expect that the guidelines provided in Section~\ref{sec:recommendations} will yield teams with much more balance than is currently typical in BUDA recreational leagues. Yet, there remains the potential to improve player ratings even further given the data at hand. Here are the most significant potential paths of research from our perspective.

\begin{itemize}

\item BUDA tracks individual game scores for all league games. This is important because in a typical BUDA recreational league, the season is 7 games long. A given team will play only 64\% of the possible competition. Under these conditions, it should be possible to use individual game scores to produce more accurate measures of team performance \cite{Langville_2012}. On the other hand, variation in individual game scores is likely to be influenced significantly by random factors such as attendance and luck. An important topic for future research is to determine the utility of individual game scores.

\item Individual game scores can also be used to measure team performance in club leagues. This could improve the predictive ability of club-assessment ratings for players in recreational leagues. The same benefits and caveats to using individual game scores apply here as well.

\item It is possible to use recreational league playing history to develop a rec-assessment rating that applies the methodology developed for club-assessment rating to recreational league data. The challenge with this approach is that these teams are designed to be comparable in skill level, so the signal-to-noise is lower. The best approach may be to consider which seasons a player has played in, since some seasons are indicative of stronger skill (winter indoor leagues) whereas others indicate weaker skill (summer or fall leagues).

\item Strategy in ultimate frisbee dictates that the best players on a given roster will touch the disc the most and be the most involved in play.  It may therefore be sensible to weight the skill level of the top portion of a given roster more than bottom portion when determining a team rating from a collection of individual ratings. This could lead to stronger correlation between team rating and team performance. However, it is likely to have the same impact regardless of rating approach, so we leave this for future work.

\item It may be the case that differences between club-assessment ratings or captain-assessment ratings and self-assessment ratings hold predictive power of team performance. For example, if someone rates themselves as a 75 (a very strong player), but their club-assessment rating is 50 (an average player), this could indicate someone who has an inflated opinion of their skills. This is likely to translate on the field to turnovers resulting from poor decision-making.  Including rating differentials as an additional component of player assessment may improve predictive power.

\item Machine learning competitions have shown very clearly that an ensemble of distinct model predictions (or ratings, in this case) outperform individual model predictions \cite{mlwave}. We investigated a simple arithmetic mean of club-assessment rating and captain-assessment rating as a first approach at an ensemble rating. The resulting correlation between ensemble rating and team performance was not significantly better or worse than club-assessment rating alone. There are many other ways to create ensemble model predictions, and future work should investigate whether any of these are more promising than the simple approach we have explored here.

\item Finally, this paper has focused on building the most accurate player rating system possible. However, generating balanced teams requires assigning players to each team. Future work should seek to optimize the draft process. 

\end{itemize}