\section{Recommendations for League Commissioners}\label{sec:recommendations}

The purpose of this section is to aid league commissioners who wish to use the information presented in this paper to improve parity in their recreational leagues. Our work reveals that a player's experience in club leagues is more predictive than their own self-assessment as well as the assessments of any captains they have had in the past. For this reason, our first recommendation is that leagues use club ratings when they are available. 

Getting the most out of club ratings requires that leagues continuously track and store roster data and game scores. Furthermore, club ratings will have maximum utility if there are stratified divisions within the club league, such that a player who participates in one division is clearly better or worse (on average) than a player who participates in another division. The appropriate base rating for each division should be tailored to each league. We recommend using the Elo system as we have done here, and then translating the resulting Elo rating into the rating system in use by the league in question.

Discuss significance of five points of club rating on a 16 player team.  Corresponds to 80 points total, or roughly replacing worst player on roster with one of best players in the league.

Recommend that league organizers use club data if available.  Organizers should strongly encourage captains to rate players (perhaps provide a monetary reward).