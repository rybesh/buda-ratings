\section{Future Work}\label{sec:summary}

We have established that individual self-assessment ratings should be used as a last resort. We have also shown that a player's club rating (based on their experience in club leagues) is more predictive of their true skill level than a player's captain rating (which itself is much better than a self-assessment rating). We expect that the guidelines provided in Section~\ref{sec:recommendations} will yield teams with much more balance than is currently typical in BUDA recreational leagues. Yet, there remains the potential to improve player ratings even further given the data at hand. Here are the most significant potential paths of research from our perspective.

\begin{itemize}

\item BUDA tracks individual game scores for all league games. This is important because in a typical BUDA recreational league, the season is 7 games long. A given team will play only 64\% of the possible competition. Under these conditions, it should be possible to use individual game scores to produce more accurate measures of team performance \cite{Langville_2012}. On the other hand, variation in individual game scores is likely to be influenced significantly by random factors such as attendance and luck. An important topic for future research is to determine the utility of individual game scores.

\item Use recreational league playing history.

\item Weight top quarter of roster more than bottom three-quarters.

\item Ensemble rating: weighted average of captain rating and club rating.

\item Test if overrating oneself relative to club rating or captain rating indicates further penalties on club rating or captain rating are needed.

\item How to run the draft (and deal with baggage).

\end{itemize}