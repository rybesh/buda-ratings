\section{Self-Assessment Ratings and Captain Ratings}\label{sec:self_rating}

One of the key components of the BUDA player rating system is self-assessment ratings from the players themselves.  These are generated from survey answers that players provide when signing up for a recreational league with BUDA. Survey questions address each player's ability to throw, catch, and run.  There is also a question related to each player's experience level, but it is given little weight in the final self-assessment rating. The majority of the weight is given to questions on the topic of throwing. Self-assessment ratings are scaled to range from 10 (lowest) to 100 (highest).

Figure~\ref{fig:self_rating} shows team performance as a function of self-assessment rating. Each dot on this plot represents one team out of the 171 spring recreational league teams in the past seven years of BUDA.  Team performance is measured using Plus/Minus per Game (for the same reasons described at the beginning of Section~\ref{sec:existing_performance}). The self-assessment rating for each team is the average of the 16 individual self ratings for that team.

The most striking result from this plot is that teams with high self-assessment ratings tend to do worse than teams with low self-assessment ratings. This statement can be quantified using the Pearson's $r$ correlation coefficient. The value of $r = -0.13$ (printed in the lower left corner of the panel) indicates a weak negative correlation between self-assessment rating and team performance.   The p-value of 0.08 (printed in the lower right corner) indicates a \approx8\% chance that the observed correlation could happen randomly. Whether or not the negative correlation is random, it is clear that there is no positive correlation between these variables.  Self-assessment ratings have limited utility in terms of predicting team performance.

BUDA has long recognized the inconsistency in self-assessment ratings.  In response, BUDA has implemented a captain rating system in addition to self-assessment ratings.  For each player on their team, captains are asked to answer the same survey questions used in the self-assessment rating.  Captain ratings should be more reliable than self-assessment ratings because captains are rating their players in a relative sense (how good is player X compared to the average player on my team?), whereas individuals are rating themselves in an absolute sense (how good am I overall?). In addition, captains tend to have more experience with league play and therefore have a better baseline from which to judge players. This is especially true in comparison to newer players that have little or no league play experience.

Describe figure showing captain ratings.

Although BUDA recognized that self-assessment ratings were inconsistent, they have underestimated how inconsistent they are. This is evidenced by the choice to use the average of the self-assessment rating and the captain rating as the overall rating to be used when drafting teams in recreational leagues.  Describe figure showing existing BUDA rating.

Note the fraction of players on a given team that do not have captain rating.
