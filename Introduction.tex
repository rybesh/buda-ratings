\section{Introduction}

Almost 5 million people in the Unitied States participate in the sport of Ultimate Frisbee.  Of these, 1.3 million are core participants---meaning that they play at least 13 times each year.  This is comparable to the sports of lacrosse (0.9 million) and rugby (0.4 million) combined \cite{sfia_2016}. Moreover, USA Ultimate---the premier organization for ultimate frisbee in the United States---has seen a 60\% growth in youth participation over the past five years \cite{usau_2016}.  While this segment of the population is focused on highly competitive play, we expect it to serve as a positive indicator for future growth of the sport at all levels.

Ultimate is unique in its emphasis on fair play, which is codified directly in the rulebook as ``Spirit of the Game``.  From Section 1. Introduction, item B:

\begin{quote} Spirit of the Game. Ultimate relies upon a spirit of sportsmanship that places the responsibility for fair play on the player. Highly competitive play is encouraged, but never at the expense of mutual respect among competitors, adherence to the agreed upon rules, or the basic joy of play.
\end{quote}

This commitment to fair play helps to explain why the vast majority of core participants play in recreational leagues in which teams are assembled randomly according to skill level such that every team has an equal chance at winning---in theory.  These leagues typically have 10 - 20 teams with 15 - 20 players on each team.  They run throughout the year, with colder climates using indoor facilities during winter.

Another manifestation of Ultimate's ``Spirit of the Game'' is the emphasis on individual assessment when determining skill level of the participants in a recreational league.  

What is BUDA?

Why spring season?

Why the past 7 years?