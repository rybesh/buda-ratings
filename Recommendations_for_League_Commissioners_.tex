\section{Recommendations for League Commissioners}\label{sec:recommendations}

The purpose of this section is to aid league commissioners who wish to use the information presented in this paper to improve parity in their recreational leagues. Our work reveals that a player's experience in club leagues, where there is a vetting process to determine if a player can participate at their preferred division of competition, is more predictive than their own self-assessment as well as the assessments of any captains they have had in the past. For this reason, our first recommendation is that leagues use club ratings when they are available. 

Getting the most out of club ratings requires that leagues continuously track and store roster data and game scores. Furthermore, club ratings will have maximum utility if there are stratified divisions within the club league, such that a player who participates in one division is clearly better or worse (on average) than a player who participates in another division. The appropriate base rating for each division should be tailored to each league. We recommend using the Elo system as we have done here, and then translating the resulting Elo rating into the rating system in use by the league in question. The translation process should also be tailored to each league. To aid league commissioners, the code we have used to build the club ratings for BUDA is publicly available \cite{shanegit}.

Captain ratings should be used when club ratings are not present. This is a key result for regions that do not have the participation numbers to run club leagues like BUDA. For these smaller organizations that run only recreational leagues, we highly recommend that captains regularly rate the players on their recreational league teams. Captain ratings are significantly more predictive of true skill level and future team performance than self-assessment ratings. League commissioners might even consider providing a financial reward to captains that complete player assessments at the end of each season, perhaps a league fee refund or discount for future league participation. 

Finally, it is inevitable that some fraction of the ultimate frisbee population will not have a club rating or a captain rating. For these players, we recommend the use of self-assessment ratings.