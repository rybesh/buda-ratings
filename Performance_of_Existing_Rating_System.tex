\section{Performance of Existing BUDA Rating System}\label{sec:existing_performance}

To investigate the performance of the existing player rating system used by BUDA, we use average per-game score differential, which we call ``Average Plus/Minus per Game.'' There are many justifiable ways to rate team performance \cite{Langville_2012}.  We chose average score differential for two reasons.  One is that these BUDA recreational leagues only have seven games per season, so using won-loss records leaves the sample size very small.  There are typically 15-20 goals scored per game, so using score differential increases the sample size significantly.  The second reason is more philosophical in nature.  For the purposes of a recreational league, it is our belief that a team that loses every game by only one goal has a better experience than a team that wins one game out of seven but loses on average by four goals.  

Figure~\ref{fig:buda_performance} shows the observed distribution (blue histogram) of per-game score differential from seven years of BUDA spring recreational league data.  The full dataset comprises 171 teams.  Of these, 16 teams have Average Plus/Minus per Game above +5 while 14 teams register below -5.  

To understand whether these numbers are consistent with expectations for a balanced league, we simulate games between equally skilled teams and monitor the simulated score differentials.  BUDA spring league games last 70 minutes.  Teams combine to score 18 goals per game, on average. This implies a goal-scoring rate of 0.26 goals/minute, or 0.13 goals/minute for a single team.  BUDA does not collect data on when goals are scored, so we assume that goal-scoring in Ultimate follows a Poisson distribution, similar to soccer \cite{Heuer_2010}.

The simulation includes 171 teams, each of which has 7 games against an equally skilled team---i.e., both teams score goals according to a Poisson distribution with an expectation value of 0.13 goals/minute. The resulting distribution of average per-game score differential is represented by the grey histogram in Figure~\ref{fig:buda_performance}.  In the simulation, there are exactly zero teams with per-game score differentials above +5 or below -5.  The simulation also results in fewer teams with plus/minus averages above +3 or below -3 compared to the observed BUDA distribution.  In contrast, there are many more simulated teams with plus/minus averages close to zero compared to the observed BUDA distribution.  This conclusively demonstrates that teams in BUDA recreational leagues are significantly imbalanced.

To give a concrete understanding of the skill disparity needed to lose by an average of 5 goals per game, we include Figure~\ref{fig:buda_performance}.  This uses the same simulations referenced earlier in this section to show the percent chance that a given team has of winning a game versus a team against which it is evenly matched (blue curve) and against which it is expected to lose by 5 goals per game (green curve). We simulate unbalanced teams by adjusting the scoring rate of the weaker team down by 3.5 goals over 70 minutes (for a goal scoring rate of 0.08 goals/minute) and the stronger team up by 1.5 goals over 70 minutes (for a goal scoring rate of 0.15 goals/minute).

Consider first the case of equally skilled teams.  At the beginning of the game, the evenly matched teams both have a 44.7\% chance of winning (chance of a tie is 10.6\%). As time remaining decreases, the chance of a tie increases gradually. With 10 minutes remaining, the chance of a tie begins to increase dramatically, but each team still remains equally likely to win.

Next, the unbalanced teams.  The team that expects to lose by five starts the game with only a 8.5\% chance at winning.  By halftime, the weaker team expects to be down by 2.5 goals, so the chance of winning has decayed to only 3.4\%.  With 10 minutes remaining, the chance of winning is negligible. This emphasizes that it is very significant to lose games by an average of 5 goals per game.