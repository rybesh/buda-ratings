\section{Self-Assessment Ratings and Captain Ratings}\label{sec:self_rating}

One of the key components of the BUDA player rating system is self-assessment ratings from the players themselves.  These are generated from survey answers that players provide when signing up for a recreational league with BUDA. Survey questions address each player's ability to throw, catch, and run.  There is also a question related to each player's experience level, but it is given little weight in the final self-assessment rating. The majority of the weight is given to questions on the topic of throwing. Self-assessment ratings are scaled to range from 10 (lowest) to 100 (highest).

BUDA has long recognized the inconsistency in self-assessment ratings.  In response, BUDA has implemented a captain rating system in addition to self-assessment ratings.  For each player on their team, captains are asked to answer the same survey questions used in the self-assessment rating.  It is sensible to expect that captain ratings will be more reliable than self-assessment ratings because captains are rating their players in a relative sense (how good is player X compared to the average player on my team?), whereas individuals are rating themselves in an absolute sense (how good am I overall?). In addition, captains tend to have more experience with league play and therefore have a better baseline from which to judge players. This is especially true in comparison to newer players that have little or no league play experience.

Explain what BUDA draft rating is.

Note the fraction of players on a given team that do not have captain rating.

Correlation between self-assessment rating and average per game goal differential.

Correlation between captain rating and average per game goal differential.

Fraction of players in a league that have a captain's rating.