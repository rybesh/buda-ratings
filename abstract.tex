Close to 1.4 million people in the United States play the sport of ultimate frisbee at least 13 times annually. Many of these core participants play in recreational leagues that attempt to build balanced teams by assigning players according to skill level. Historically, skill level has been determined by each player’s self-assessment. Such ratings tend to be biased and create significantly imbalanced teams. In this contribution, we use seven years of recreational ultimate frisbee league data from one of the largest ultimate frisbee organizations in the world, the Boston Ultimate Disc Alliance. We show that a rating system based on self-assessment regularly results in 8\% of teams in a given league losing more than 85\% of their games---approximately double the rate expected if teams were equal in skill level. We introduce an alternative method for rating players based on their playing history, which builds a more accurate measure of their ability and a better forecast of their team’s success. We provide guidelines for organizers to follow when incorporating this data-driven rating system into their recreational ultimate frisbee leagues.