\href{http://}{}\section{Introduction}\label{sec:introduction}

Almost 5 million people in the Unitied States participate in the sport of Ultimate Frisbee.  Of these, 1.3 million are core participants---meaning that they play at least 13 times each year.  This is comparable to the sports of lacrosse (0.9 million) and rugby (0.4 million) combined \cite{sfia_2016}. Moreover, USA Ultimate---the premier organization for ultimate frisbee in the United States---has seen a 60\% growth in youth participation over the past five years \cite{usau_2016}.  While this segment of the population is focused on highly competitive play, we expect it to serve as a positive indicator for future growth of the sport at all levels.

Ultimate is unique in its emphasis on fair play, which is codified directly in the rulebook as "Spirit of the Game".  From Section 1, Introduction, item B:

\begin{quote} Spirit of the Game. Ultimate relies upon a spirit of sportsmanship that places the responsibility for fair play on the player. Highly competitive play is encouraged, but never at the expense of mutual respect among competitors, adherence to the agreed upon rules, or the basic joy of play.
\end{quote}

This commitment to fair play helps to explain why many core participants play in recreational leagues in which teams are assembled randomly according to skill level such that every team has an equal chance at winning---in theory.  These leagues typically have 10 - 20 teams with 15 - 20 players on each team.  They run throughout the year, with colder climates using indoor facilities during winter.

Another manifestation of Ultimate's ``Spirit of the Game'' is the emphasis on individual assessment when determining skill level of the participants in a recreational league.  A robust measure of skill level is critical to creating balanced teams.  Historically, teams in recreational leagues have been created by ranking players according to their self-assessed skill level and then assigning them to teams according to a snake-like draft system in which the order reverses with each round.  The team that draft firsts in round 1 drafts last in round 2, then first in round 3, etc.

One of the goals of this paper is to investigate the performance of self-assessment ratings relative to what is expected for teams of equal skill level.  Our results show the poor performance of self-assessment ratings and have motivated us to undertake a second goal: identify an alternative rating system that makes use of data on what teams players have played on previously. 

We make use of seven years of data from the Boston Ultimate Disc Alliance (BUDA) in order to achieve our goals. BUDA is a non-profit organization run by volunteers to give people in the Boston area the opportunity to play Ultimate. BUDA organizes nearly 100\% of the leagues in the greater Boston area. BUDA has collected data on who has played on what team and how that team performed during the season.  This is the data that is used to compute club ratings in Section~\ref{sec:club_rating}.  BUDA also has stored self-assessment ratings and captain ratings over the years.  The excellent BUDA database is what makes the research in this paper possible.  

In this paper, we focus on data during the spring season.  There are five seasons of BUDA recreational league play in a single year: one each in spring, summer, and fall, and two that take place indoors (during the long New England winter.) Spring is an ideal testbed for this analysis because the player pool exhibits the widest range of skill level compared to the other seasons.  In summer and fall, the best players need practice time with their club teams, so they skip recreational league.  In winter, the cost to play is higher and the distance to the facility is greater.  Even more important, the playing field is a converted hockey rink---it is much smaller than a normal field for Ultimate and has walls along the edge that must be avoided when playing.  For these reasons, only the most committed players continue to play in winter time. This skews the average skill level of winter leagues much higher than the outdoor leagues.  We expect that the results of this paper will apply to the non-spring recreational leagues, but testing exactly how much so is a topic for future work.

BUDA has collected data since 2001. In this paper, we focus on the most recent seven years of data, from 2010 through 2016.  Prior to 2010, the fraction of the player population in a given league that had a captain rating was too low to provide a significant benefit compared to self-assessment ratings.  Similarly, the data needed to compute a club rating for each player was too sparse prior to 2010. By focusing on data after 2010, we ensure that the captain ratings and club ratings are a fair representation of their ability to forecast future performance.

The outline of our paper is as follows. In Section~\ref{sec:existing_performance}, we quantify the performance of the existing BUDA rating system relative to expectations for equally skilled teams. Section~\ref{sec:self_rating} shows that most of the poor performance of the existing system is due to its reliance on self-assessment ratings.  This is achieved via a comparison of the predictive power of self-assessment ratings and captain ratings regarding team performance.  Section~\ref{sec:club_rating} describes the methodology we use to compute club ratings and shows that club ratings are more predictive of team performance than either self-assessment ratings or captain ratings. In Section~\ref{sec:recommendations}, we present clear guidelines for league commissioners to follow in order to make teams in recreational ultimate frisbee leagues as balanced as possible. Finally, in Section~\ref{sec:summary}, we summarize our findings and highlight goals of future work.