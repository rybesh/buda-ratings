\section{Performance of Existing BUDA Rating System}\label{sec:existing_performance}

To investigate the performance of the existing player rating system used by BUDA, we use average per-game score differential, which we call ``Average Plus/Minus per Game''. There are many justifiable ways to rate team performance \cite{Langville_2012}.  We chose average score differential for two reasons.  One is that BUDA recreational league have seven games per season, so using won-loss record leaves the sample size very small.  There are typically 15-20 goals scored per game, so using score differential increases the sample size significantly.  The second reason is more philosophical in nature.  For the purposes of a recreational league, it is our belief that a team that loses every game by only one goal has a better experience than a team that wins one game out of seven but loses on average by four goals.  

Figure~\ref{fig:buda_performance} shows the observed distribution (blue histogram) of per-game score differential from seven years of BUDA spring recreational league data.  This figure also shows the expected distribution in a league where teams were all equally skilled (grey histogram). The full dataset comprises 171 teams.  Of these, 16 teams have Average Plus/Minus per Game above +5 while 14 teams register below -5.  Exactly zero such teams would be expected if the teams were equally skilled.

To give a concrete understanding of just how bad a team is that loses on average by 5 goals per game, we include Figure~\ref{fig:buda_performance}.  This shows the percent chance that a given team has of winning a game versus a team against which it is evenly matched (blue curve) and against which it is expected to lose by 5 goals per game (green curve). 

Consider first the case of equally skilled teams.  At the beginning of the game, the evenly matched teams both have a 44.7\% chance of winning (chance of a tie is 10.6\%). As time remaining decreases, the chance of a tie increases gradually. With 10 minutes remaining, the chance of a tie begins to increase dramatically, but each team still remains equally likely to win.

Next, the unbalanced teams.  The team that expects to lose by five starts the game with only a 8.5\% chance at winning.  By halftime, the weaker team expects to be down by 2.5 goals, so the chance of winning has decayed to only 3.4\%.  With 10 minutes remaining, the chance of winning is negligible.