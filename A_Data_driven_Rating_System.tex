\section{A Data-driven Rating System}\label{sec:club_rating}

As an alternative to direct human assessment ratings (whether self-assessment or captain assessment), we advocate here for a data-driven rating system that considers each player's history within BUDA club leagues. Unlike recreational leagues in which players are randomly assigned to teams according to skill level, players in club leagues form their own teams and select a division of competition which they believe to be appropriate to their team-wide skill level.  There are four divisions ranging from casual (Division 4) to highly competitive (Division 1).  

Each club team uses a vetting process to determine whether or not a player can join their team. This is often as simple as a friend's recommendation, but it can also be as involved as a multi-day tryout. The key point is that in order to join a team, a player must have exhibited evidence that they will be competent at the skill level of the division in which that team competes.

Based on our experience of playing in different BUDA club league divisions, we begin by assigning the following ratings to each division in the style of the Elo system (reference). Division 1: 1800, Division 2: 1400, Division 3: 1100, Division 4: 900.  The idea is that a Division 1 team is expected to beat a Division 2 team 90\% of the time, a Division 2 team wins 85\% of the time against a Division 3 team, and a Division 3 team beats a Division 4 team 75\% of the time.

The next step is to acknowledge that within each division, there is a wide range of team-wide skill level. Again, our experience with BUDA suggests that a weak Division 1 team is comparable to a strong division 2 team.  With this in mind, we calculate team ratings as the following:

\begin{equation}
{\rm rating}_{\rm team} = {\rm rating}_{\rm division} + 60 \times {\rm AGD},
\end{equation}

\noindent where ${\rm AGD}$ is the average goal differential per game for that team.  Under this definition, a Division 1 team that loses on average by 3.3 goals per game has an equal rating to a Division 2 team that wins on average by 3.3 goals per game.

This methodology yields a team rating for every club team in the BUDA database.  The club rating for a given player at a given point in time is the arithmetic mean of all of the team ratings that player has played on leading up to that point in time.  Here is a concrete example.  Suppose a player has experience on an average Division 1 team (${\rm AGD} = 0$), a strong Division 2 team (${\rm AGD} = 3$), and an average Division 2 team (${\rm AGD} = 0$).  In this scenario, their club rating will be $\frac{1}{3}(1800 + 1580 + 1400) = 1593$.

The last step that is unique to assigning a club rating to a player is the conversion to the same 0 to 100 scale used by BUDA for self-assessment ratings and captain ratings.  The simple philosophy guiding this process is (once again) our own personal experience within BUDA, which suggests that an average Division 1 player will have a club rating of 80, while an average Division 2 player will have a club rating of 60. An average Division 3 player gets a club rating of 45 and an average Division 4 player gets a club rating of 30. We use interpolation to fill in the gaps for ratings between these values.

What about club players that don't have a self-rating??

Plot showing club rating vs. average goal differential per game (hat).  This is the money plot.