\section{Self-Assessment Ratings and Captain Ratings}\label{sec:self_rating}

BUDA’s recreational league teams are assembled using player ratings. The rating for a given player is derived from that player's self-assessment of skill level (self-assessment rating) as well as from a captain's assessment of skill level (captain-assessment rating). In BUDA recreational leagues, all players have a self-assessment rating, but not all have captain's assessment ratings.  In this section, we discuss each of these ratings in turn.

Self-assessment ratings are generated from survey answers that players provide when first signing up for a BUDA recreational league. Survey questions address each player's ability to throw, catch, and run.  A question related to each player's experience level is included, but it is given little weight in the final self-assessment rating. The majority of the weight is given to questions on the topic of throwing. Self-assessment ratings are scaled to range from 10 (lowest) to 100 (highest).

One of the limitations of these ratings is that players often forget to update their survey responses over time. Many of the self-assessment ratings in the BUDA database are several years old and do not reflect present injuries or significantly increased skill.

Figure~\ref{fig:correlation_self} shows points per game differential as a function of self-assessment rating. Each dot on this plot represents one team out of the 171 spring recreational league teams in the past seven years of BUDA. We use points per game differential as a proxy for team performance for the same reasons outlined at the beginning of Section~\ref{sec:existing_performance}. The self-assessment rating for each team is the average of the individual self-assessment ratings for the players on that team.

The most striking result from this plot is that teams with high self-assessment ratings tend to do worse than teams with low self-assessment ratings. This statement can be quantified using the Pearson's $r$ correlation coefficient. The value of $r = -0.13$ (printed in the lower left corner of the panel) indicates a weak negative correlation between self-assessment rating and team performance.   The p-value of 0.08 (printed in the lower right corner) indicates a roughly 8\% chance that the observed correlation could happen randomly. Whether or not the negative correlation is random, it is clear that there is no positive correlation between these variables.  Self-assessment ratings have limited utility in terms of predicting team performance.

BUDA has long been aware of the inconsistency of self-assessment ratings.  In response, BUDA has implemented captain-assessment ratings.  For each player on their team, captains are asked to answer the same survey questions used in the self-assessment rating.  Captain ratings should be more reliable than self-assessment ratings because captains are rating their players in a relative sense (how good is player X compared to the average player on my team?), whereas individuals are rating themselves in an absolute sense (how good am I overall?). In addition, captains tend to have more experience with league play and therefore have a better baseline from which to judge players. This is especially true in comparison to newer players that have little or no league play experience.

Figure~\ref{fig:correlation_captain} shows how points per game differential relates to captain rating.  As before, a single dot on this diagram is one team from the past seven years of BUDA spring recreational league.  The captain-assessment rating for a given team is the average of the captain-assessment ratings for the players on that team. When a captain-assessment rating is not available for a given player, that player's self-assessment rating is used instead (more on this at the end of this section).

There is a strong, positive correlation between captain-assessment rating and team performance---if the captain rating indicates a team will do well, they do well (and vice versa). The Pearson's correlation coefficient is $r = 0.31$, and the p-value of $<0.01$ indicates a very low probability of this result occurring by chance. There remains a broad range of team performance for a given captain rating, but some of this is to be expected given the inherent uncertainty in points per game differential, even between two teams of equal skill level (as shown in Section~\ref{sec:existing_performance}).

The rating system currently in place in BUDA recreational leagues is the arithmetic mean of the self-assessment rating and the captain-assessment rating.  For example, if a player has a self-assessment rating of 40 and a captain-assessment rating of 70, their BUDA rating will be 55. Figure~\ref{fig:correlation_buda} shows the relationship between team performance and current BUDA rating. There is a weak positive correlation ($r = 0.10$), but the p-value of 0.21 indicates a roughly 1 in 5 chance that this result could happen randomly.  In other words, the existing BUDA rating is not predictive of team performance. This result shows that imbalanced teams in recreational leagues are due to imperfections in the rating system rather than the draft process itself.

As a final note to this section, it is worthwhile to comment further on captain-assessment ratings. BUDA asks captains to rate the players on their teams, but the role of the captain is a volunteer position. From the captain's perspective, there is no monetary reward for going the extra mile and rating all of the players on one's team——typically around 16 players. Consequently, many captains abdicate their responsibility of rating the players on their team. As a result, many players in the BUDA database have no captain rating, despite playing for years in BUDA. On an average roster of 16 players, 4 players will not have a captain rating. In some cases, as many as 8 players lack a captain-assessment rating. This limits the utility of captain-assessment ratings and motivates the need for an alternative, data-driven rating system that does not rely on continual human intervention to function properly.
