\section{A Data-driven Rating System}\label{sec:club_rating}

As an alternative to direct human assessment ratings (whether self-assessment or captain assessment), we advocate here for a data-driven rating system that considers each player's history within BUDA club leagues. Unlike recreational leagues in which players are randomly assigned to teams according to skill level, players in club leagues form their own teams and select a division of competition which they believe to be appropriate to their team-wide skill level.  There are four divisions ranging from casual (Division 4) to highly competitive (Division 1).  

Each club team uses a vetting process to determine whether or not a player can join their team. This is often as simple as a friend's recommendation, but it can also be as involved as a multi-day tryout. The key point is that in order to join a team, a player must have exhibited evidence that they will be competent at the skill level of the division in which that team competes.

Based on our experience of playing in different BUDA club league divisions, we begin by assigning the following ratings to each division in the style of the Elo system (reference). Division 1: 1800, Division 2: 1400, Division 3: 1100, Division 4: 900.  The idea is that a Division 1 team is expected to beat a Division 2 team 90\% of the time and similarly for Division 2 vs. Division 3.  Division 3 and Division 4 are somewhat closer in skill level

Definition of club rating.  Math behind computation.

Plot showing club rating vs. average goal differential per game (club).

What about club players that don't have a self-rating??

Plot showing club rating vs. average goal differential per game (hat).  This is the money plot.