Nearly 1.5 million people in the United States are core participants in the sport of ultimate frisbee, meaning that they play at least 12 times annually. The vast majority of core participants play in recreational leagues in which teams are formed by randomly assigning players according to skill level such that league-wide parity is maximized. Historically, skill level has been determined by each individual’s self-assessment. Such ratings tend to be biased in ways that create significantly imbalanced teams. In this contribution, we use seven years of recreational ultimate frisbee league data from the Boston Ultimate Disc Alliance to show that a rating system based on self-assessment regularly results in 8\% of teams in a given league losing more than 85\% of their games---approximately double the rate expected if teams were equal in skill level.  We introduce an alternative method for rating players that makes use of information regarding which club teams a player has joined in the past. Because club teams include a vetting process for selecting players, this new rating system is more predictive of future performance than self-ratings. 